\documentclass[finnish,russian,]{article}
\usepackage[T1]{fontenc}
\usepackage{lmodern}
\usepackage{amssymb,amsmath}
\usepackage{ifxetex,ifluatex}
\usepackage{fixltx2e} % provides \textsubscript
% use upquote if available, for straight quotes in verbatim environments
\IfFileExists{upquote.sty}{\usepackage{upquote}}{}
\ifnum 0\ifxetex 1\fi\ifluatex 1\fi=0 % if pdftex
  \usepackage[utf8]{inputenc}
\else % if luatex or xelatex
  \ifxetex
    \usepackage{mathspec}
    \usepackage{xltxtra,xunicode}
  \else
    \usepackage{fontspec}
  \fi
  \defaultfontfeatures{Mapping=tex-text,Scale=MatchLowercase}
  \newcommand{\euro}{€}
\fi
% use microtype if available
\IfFileExists{microtype.sty}{\usepackage{microtype}}{}
\ifxetex
  \usepackage[setpagesize=false, % page size defined by xetex
              unicode=false, % unicode breaks when used with xetex
              xetex]{hyperref}
\else
  \usepackage[unicode=true]{hyperref}
\fi
\hypersetup{breaklinks=true,
            bookmarks=true,
            pdfauthor={Juho Härme},
            pdftitle={Aspektien erityismerkitykset},
            colorlinks=true,
            citecolor=blue,
            urlcolor=blue,
            linkcolor=magenta,
            pdfborder={0 0 0}}
\urlstyle{same}  % don't use monospace font for urls
\setlength{\parindent}{0pt}
\setlength{\parskip}{6pt plus 2pt minus 1pt}
\setlength{\emergencystretch}{3em}  % prevent overfull lines
\setcounter{secnumdepth}{5}
\ifxetex
  \usepackage{polyglossia}
  \setmainlanguage{russian}
\else
  \usepackage[finnish,russian]{babel}
\fi

\title{Aspektien erityismerkitykset}
\author{Juho Härme}
\date{2015-12-12}

\begin{document}
\maketitle

\section{Aspektien
erityismerkitykset}\label{aspektien-erityismerkitykset}

Apekti- ja liikeverbiteoria \textgreater{} Aspekti Aspektin sijaintiin
vaikuttavat tekijät \textgreater{} Pragmaattiset tekijät \textgreater{}
Konteksti venäjänkielinen termi: частные видовые значения

\subsection{Määritelmä}\label{muxe4uxe4ritelmuxe4}

Sen lisäksi, että aspektologiassa erotetaan aspektien invariantit
merkitykset, voidaan sekä imperfektiivinen että perfektiivinen aspekti
jakaa edelleen tapausryppäisiin, jotka muistuttavat merkitykseltään ja
käyttötilanteiltaan toisiaan. Jokainen tällainen tapausrypäs,
erityismerkitys, heijastaa omalla tavallaan ko. aspektille tyypillisiä
semanttisia piirteitä kuten toiminnan sisäistä rajaa, sidonnaisuutta
aikaan tai prosessuaalisuutta. Kuitenkin jotkin piirteet tai
ominaisuudet tulevat selvemmin esille kuin toiset.

Ei ole olemassa tarkkaa määritelmää tai yksimielisyyttä siitä, mitä
kaikkia erityismerkityksia aspekteilla on. Esimerkiksi V.S.Bondarko
(2005: 238--246) erottaa kaikkiaan kuusi imperfektiivisen ja neljä
perfektiivisen aspektin erityismerkitystä, kun taas Nikunlassi (2002:
178--187) luettelee peräti yhdeksän erityismerkityksen pääluokkaa, jotka
lisäksi jakautuvat muutamiin alaluokkiin. Aspektologian klassikko Ju. S.
Maslov (1959) puolestaan erottaa kummallekin aspektille kolme tärkeintä
erityismerkitystä.

Mainittujen luokitusten erot johtuvat pääosin siitä, että toisissa
analyyseissa jokin erityismerkitys jaetaan useaksi alamerkityksesi kun
taas toisissa käytetään laveampaa luokittelua. Ei toki ole niin, että
esimerkiksi Maslovin suppeassa luokituksessa jätettäisiin jotkin
merkitykset kokonaan huomiotta: ne vain ryhmitellään osaksi laajempaa
kokonaisuutta kuin muissa luokituksissa.

Tällä kurssilla keskitytään vain valikoituun määrään erityismerkityksiä,
niin että painotus on käytetyimmissä ja laajimmin sovellettavissa
olevissa luokissa.

\subsection{Perfektiivisen aspektin
erityismerkityksiä}\label{perfektiivisen-aspektin-erityismerkityksiuxe4}

Erityismerkityksiä käsiteltäessä on usein tapana aloittaa perfektistä,
jolla erityismerkityksiä on perinteisesti nähty vähemmän kuin
imperfektiivisellä aspektilla. Tässä yhteydessä käsitellään seuraavia
perfektiivisen aspektin merkitysryppäitä:

\begin{enumerate}
\def\labelenumi{\arabic{enumi}.}
\itemsep1pt\parskip0pt\parsep0pt
\item
  konkreettis-faktuaalinen merkitys (конкретно-фактическое значение)
\item
  havainnollisen esimerkin merkitys (наглядно-примерное значение)
\item
  perfektin merkitys (перфектное значение)
\item
  potentiaalinen merkitys (потенциальное значение)
\item
  summatiivinen merkitys (суммарное значение)
\end{enumerate}

\subsubsection{Konkreettis-faktuaalinen
merkitys}\label{konkreettis-faktuaalinen-merkitys}

\begin{enumerate}
\def\labelenumi{(\arabic{enumi})}
\itemsep1pt\parskip0pt\parsep0pt
\item
  -- Отпустили тебя? -- отпустили.\\
\item
  Скоро вернется. Пройдите, подождите.\\
\item
  Он повторил мне свой вопрос.\\
\item
  Он написал письмо и вложил его в конверт
\end{enumerate}

Konkreettis-faktuaalista merkitystä voi pitää perfektiivisen aspektin
\emph{perusmerkityksenä}, koska

\begin{enumerate}
\def\labelenumi{\alph{enumi}.}
\itemsep1pt\parskip0pt\parsep0pt
\item
  Se on vähiten riippuvainen kontekstista
\item
  Se ilmaisee selkeimmin perfektiiviselle aspektille tyypillisiä
  piirteitä
\item
  Se on laajimmin levinnyt
\end{enumerate}

Nimensä mukaisesti konkreettis-faktuaalinen erityismerkitys viittaa
toimintaan, joka on

\begin{enumerate}
\def\labelenumi{\arabic{enumi}.}
\itemsep1pt\parskip0pt\parsep0pt
\item
  Konkreettista:

  \begin{itemize}
  \itemsep1pt\parskip0pt\parsep0pt
  \item
    kuvaa yhtä yksittäistä, konkreettista toimintakertaa\\
  \end{itemize}
\item
  Faktuaalista

  \begin{itemize}
  \itemsep1pt\parskip0pt\parsep0pt
  \item
    kuvaa toiminnan totaalisena
  \end{itemize}
\end{enumerate}

\subsubsection{Potentiaalinen merkitys}\label{potentiaalinen-merkitys}

\begin{enumerate}
\def\labelenumi{(\arabic{enumi})}
\setcounter{enumi}{4}
\itemsep1pt\parskip0pt\parsep0pt
\item
  Ему можно доверять. Он всегда поможет другу.
\item
  Он не сдаст экзамен, слишком много пропускал занятия
\end{enumerate}

Potentiaalinen merkitys kuvaa puhujan arvioita tekijän kyvystä suorittaa
tehtävä.

\subsubsection{Perfektin merkitys}\label{perfektin-merkitys}

\begin{enumerate}
\def\labelenumi{(\arabic{enumi})}
\setcounter{enumi}{6}
\itemsep1pt\parskip0pt\parsep0pt
\item
  Он похудел и постарел
\end{enumerate}

Muodostaa yhteyden nykyhetkeen.

\subsection{Imperfektiivisen aspektin
erityismerkityksiä}\label{imperfektiivisen-aspektin-erityismerkityksiuxe4}

Luokittelusta riippumatta imperfektiivisen aspektien erityismerkityksiä
on perinteisesti erotettu useampia kuin perfektiivisellä aspektilla.
Tässä keskitytään seuraaviin:

\begin{enumerate}
\def\labelenumi{\arabic{enumi}.}
\itemsep1pt\parskip0pt\parsep0pt
\item
  konkreettis-prosessuaalinen merkitys (конкретно-процессное значение)
\item
  rajoittamattoman toiston merkitys (неограниченно-кратное значение)
\item
  yleisesti toteava merkitys (обобщенно-фактическое значение)
\item
  rajoitetun toiston merkitys (ограниченно-кратное значение)
\item
  Tilaan, ominaisuuteen tai suhteeseen liittyvä merkitys (значение
  неактуального осостояния, свойства или отношения)
\end{enumerate}

\subsubsection{Konkreettis-prosessuaalinen
merkitys}\label{konkreettis-prosessuaalinen-merkitys}

\begin{enumerate}
\def\labelenumi{(\arabic{enumi})}
\setcounter{enumi}{7}
\itemsep1pt\parskip0pt\parsep0pt
\item
  Когда мама пришла домой, я играл на фортепиано
\item
  Молодая женщина сидела у окна вагона и читала (ras)
\item
  Я слышу шаги (nik)
\end{enumerate}

Konkreettis-faktuaalista merkitystä voi pitää imperfektiivisen aspektin
\emph{perusmerkityksenä}, koska

\begin{enumerate}
\def\labelenumi{\alph{enumi}.}
\itemsep1pt\parskip0pt\parsep0pt
\item
  Se on vähiten riippuvainen kontekstista
\item
  Se ilmaisee selkeimmin imperfektiiviselle aspektille tyypillisiä
  piirteitä
\item
  Se on laajimmin levinnyt
\end{enumerate}

Konkreettis-prosessuaalista merkitystä välittävä toiminta on:

\begin{enumerate}
\def\labelenumi{\arabic{enumi}.}
\itemsep1pt\parskip0pt\parsep0pt
\item
  Konkreettista:

  \begin{itemize}
  \itemsep1pt\parskip0pt\parsep0pt
  \item
    kuvaa yhtä yksittäistä, konkreettista toimintakertaa\\
  \end{itemize}
\item
  Prosessuaalista

  \begin{itemize}
  \itemsep1pt\parskip0pt\parsep0pt
  \item
    ei kuvaa toiminnan totaalisena
  \item
    ei erota toiminnalle sisäistä rajaa
  \end{itemize}
\end{enumerate}

\paragraph{Alamerkityksiä}\label{alamerkityksiuxe4}

\subparagraph{Konatiivinen merkitys}\label{konatiivinen-merkitys}

\begin{enumerate}
\def\labelenumi{(\arabic{enumi})}
\setcounter{enumi}{10}
\itemsep1pt\parskip0pt\parsep0pt
\item
  Вот тебе и муж предоставлен. Убивали, да не убили (Bon)
\item
  Я давал ему деньги, но он не брал их
\end{enumerate}

Konatiivisessa merkityksessä viitataan yritykseen suorittaa jokin
toiminto

\subparagraph{Kestoa korostava
merkitys}\label{kestoa-korostava-merkitys}

\begin{enumerate}
\def\labelenumi{(\arabic{enumi})}
\setcounter{enumi}{12}
\itemsep1pt\parskip0pt\parsep0pt
\item
  Значит, опять всю ночь будет выть собака. (Bon)
\item
  Ваня все читал и читал
\item
  Студенты занимались с утра до вечера
\end{enumerate}

Painotetaan toiminnan kestoa esimerkiksi verbin toistolla tai kestoa
korostavilla adverbiaaleilla kuten всю ночь, долго, целый день

BONDARKO, A. 2005. \emph{Теория морфологических категории и
аспектологические исследования}. Moskova: Jazyki slavjanskoj kultury.

MASLOV, J.S. 1959. \emph{Вопросы грамматики в болгарском литературном
языке}. Москва.

NIKUNLASSI, A. 2002. \emph{Johdatus venäjän kieleen ja sen
tutkimukseen}. Helsinki: Finn Lectura.

\end{document}
